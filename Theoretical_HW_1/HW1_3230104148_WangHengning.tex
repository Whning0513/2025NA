\documentclass[a4paper]{article}
\usepackage[affil-it]{authblk}
\usepackage[backend=bibtex,style=numeric]{biblatex}
\usepackage{amsmath}
\usepackage{geometry}
\geometry{margin=1.5cm, vmargin={0pt,1cm}}
\setlength{\topmargin}{-1cm}
\setlength{\paperheight}{29.7cm}
\setlength{\textheight}{25.3cm}

\addbibresource{citation.bib}

\begin{document}
% =================================================
\title{Numerical Analysis homework \# 1}
% 第一次
\author{Wang Hengning 3230104148
  \thanks{Electronic address: \texttt{whning@zju.edu.cn}}}
\affil{Computational mathematics 2301, Zhejiang University }


\date{Due time: \today}

\maketitle

%\begin{abstract}
%    The abstract is not necessary for the theoretical homework, 
%    but for the programming project, 
%    you are encouraged to write one.      
%\end{abstract}


% ============================================
% \section*{I. Briefly repeat the problem}

\section*{I. Analysis of the Bisection Method on the Interval $[1.5, 3.5]$}
% II. 对区间 $[1.5, 3.5]$ 上的二分法分析

\subsection*{Width of the interval}
% 第 n 步的区间宽度

Given the initial interval $[a_0, b_0] = [1.5, 3.5]$, the initial width is $W_0 = b_0 - a_0 = 2$.
% 给定初始区间 $[a_0, b_0] = [1.5, 3.5]$,初始宽度为 $W_0 = b_0 - a_0 = 2$。
The interval width is halved at each step, so the width at step $n$, $W_n$, is:
% 区间宽度在每一步迭代中减半,因此第 $n$ 步的宽度 $W_n$ 为:
\[
W_n = \frac{W_0}{2^n} = \frac{2}{2^n} = \frac{1}{2^{n-1}}
\]

\subsection*{Supremum of the error}
% 根与区间中点的距离上确界

At step $n$, let the interval be $[a_n, b_n]$ and the midpoint $c_n = (a_n+b_n)/2$. The root $r$ satisfies $r \in [a_n, b_n]$.
% 在第 $n$ 步,设区间为 $[a_n, b_n]$,中点为 $c_n = (a_n+b_n)/2$。根 $r$ 满足 $r \in [a_n, b_n]$。
The distance $|r - c_n|$ is maximized when $r$ is located at an endpoint of the interval.
% 当根 $r$ 位于区间端点时,距离 $|r - c_n|$ 达到最大值。
This gives the supremum of the error as 1/2 the interval width.
% 由此可得误差的上确界为区间宽度的一半。
\[
\sup|r - c_n| = c_n - a_n = b_n - c_n = \frac{b_n - a_n}{2} = \frac{W_n}{2^1} = \frac{W_{n-1}}{2^2} = \frac{1}{2^n}
\]

% 222222222222222222222222222222

\section*{II. Proof for the Number of Steps for a Given Relative Error}
% II. 给定相对误差所需步数的证明

Denote the root by $r$, we need to find the number of steps $n$ such that $|r-c_n|/|r|$ is no greater than $\epsilon$.
% 我们需要找到使此误差不大于 $\epsilon$ 的步数 $n$。
\[
\frac{|r-c_n|}{|r|} \le \epsilon
\]
First, the absolute error is bounded by half the interval width at step $n$.
% 首先,我们为分子和分母建立界。绝对误差的上界为第 $n$ 步区间宽度的一半。
\[
|r - c_n| \le \frac{b_n - a_n}{2} = \frac{b_0 - a_0}{2^{n+1}}
\]
Given that $r \in [a_0, b_0]$ and $a_0 > 0$, the magnitude of the root is bounded below by $a_0$.
% 给定 $r \in [a_0, b_0]$ 且 $a_0 > 0$,根的绝对值 $|r|$ 的下界为 $a_0$。
\[
|r| \ge a_0
\]
Therefore, we can conclude that: 
% 结合这些界,可得到相对误差的一个上界。
\[
\frac{|r-c_n|}{|r|} \le \frac{(b_0 - a_0) / 2^{n+1}}{a_0} = \frac{b_0 - a_0}{a_0 \cdot 2^{n+1}}
\]
To guarantee the desired accuracy, we enforce this upper bound to be less than or equal to $\epsilon$.
% 为保证所需精度,我们强制此上界小于或等于 $\epsilon$。
\[
\frac{b_0 - a_0}{a_0 \cdot 2^{n+1}} \le \epsilon
\]
\[
\frac{b_0 - a_0}{\epsilon \cdot a_0} \le 2^{n+1}
\]
Taking the logarithm of both sides yields:
% 对两边取对数可得:
\[
\log\left(\frac{b_0 - a_0}{\epsilon \cdot a_0}\right) \le \log(2^{n+1})
\]
\[
\log(b_0 - a_0) - \log\epsilon - \log a_0 \le (n+1)\log 2
\]
Rearranging terms :
\[
n \ge \frac{\log(b_0 - a_0) - \log\epsilon - \log a_0}{\log 2} - 1
\]
This completes the proof.
% 证明完毕。


%33333333333333333333333333333333333
\section*{III. Application of Newton's Method}
% II. 牛顿法应用

The problem is to find a root for the polynomial equation $p(x) = 4x^3 - 2x^2 + 3 = 0$ with the starting point $x_0 = -1$.
% 本题为求解多项式方程 $p(x) = 4x^3 - 2x^2 + 3 = 0$ 的一个根,初始点为 $x_0 = -1$。
We have the derivative of $p(x)$:
% 首先,我们计算 $p(x)$ 的导数。
\[
p'(x) = 12x^2 - 4x
\]
The iteration formula for this specific problem is:
% 牛顿法的迭代公式为 $x_{k+1} = x_k - p(x_k)/p'(x_k)$。针对本题,该公式为:
\[
x_{k+1} = x_k - \frac{4x_k^3 - 2x_k^2 + 3}{12x_k^2 - 4x_k}
\]

Starting with $x_0 = -1$, we perform 4 iterations. The results are organized in the following table. 
% 从 $x_0 = -1$ 开始,我们执行四次迭代。结果整理在下表中。

\begin{center}
\begin{tabular}{c | c c c}
\hline
Iteration, $k$ & $x_k$ & $p(x_k)$ & $p'(x_k)$ \\
\hline
0 & -1.000000 & -3.000000 & 16.000000 \\
1 & -0.812500 & -0.465820 & 11.171875 \\
2 & -0.770804 & -0.020138 & 10.212886 \\
3 & -0.768832 & -0.000044 & 10.168568 \\
4 & -0.768828 & -0.000000 & 10.168472 \\
\hline
\end{tabular}
\end{center}

After four iterations, the approximation of the root is $x_4$.
% 经过四次迭代后,根的近似值为 $x_4$。
\[
x_4 \approx -0.768828
\]


%44444444444444444444444444
\section*{IV. Convergence Analysis of a Modified Newton's Method}
% II. 修正牛顿法的收敛性分析

Let $r$ be the root, with $f(r)=0$, and let the error be $e_n = x_n - r$.
% 设 $r$ 为真根,满足 $f(r)=0$,并设误差为 $e_n = x_n - r$。
The error recurrence relation derived from the iteration formula is:
% 从迭代公式推导出的误差递推关系为:
\[
e_{n+1} = x_{n+1} - r = e_n - \frac{f(x_n)}{f'(x_0)}
\]
Expanding $f(x_n) = f(r+e_n)$ as a Taylor series around $r$ gives:
% 将 $f(x_n) = f(r+e_n)$ 在 $r$ 附近进行泰勒级数展开,可得:
\[
f(x_n) = f'(r)e_n + \frac{f''(r)}{2}e_n^2 + O(e_n^3)
\]
% 将此展开式代入递推关系,得到误差方程。
\[
e_{n+1} = e_n - \frac{f'(r)e_n + \frac{f''(r)}{2}e_n^2 + O(e_n^3)}{f'(x_0)}
\]
\[
e_{n+1} = \left(1 - \frac{f'(r)}{f'(x_0)}\right)e_n - \left(\frac{f''(r)}{2f'(x_0)}\right)e_n^2 + O(e_n^3)
\]
By comparing with the form $e_{n+1} = Ce_n^s$, we identify the exponent of the dominant term as $s$ and its coefficient as $C$.Therefore, this iteration formula is \textbf{linearly convergent}.
% 通过与形式 $e_{n+1} = Ce_n^s$ 进行比较,我们可确定主导项的指数为 $s$,其系数为 $C$。
\[
s = 1
\]
\[
C = 1 - \frac{f'(r)}{f'(x_0)}
\]

% 55555555555555

\section*{V. Convergence of the Iteration $x_{n+1} = \tan^{-1} x_n$}
% V. 迭代式 $x_{n+1} = \tan^{-1} x_n$ 的收敛性

The iteration $x_{n+1} = \tan^{-1}x_n$ has a unique fixed point $\alpha$ that solves $\tan^{-1}\alpha = \alpha$, which is $\alpha=0$.
% 迭代式 $x_{n+1} = \tan^{-1}x_n$ 存在唯一的不动点 $\alpha$ 满足 $\tan^{-1}\alpha = \alpha$,解得 $\alpha=0$。
If $x_0 = 0$, the sequence converges trivially. Thus, we only consider the case where $x_0 \ne 0$.
% x0=0时很明显是收敛于0的,所以我们只考虑不为0的情况。
To analyze convergence, we consider the absolute value of the terms. For any $x \ne 0$, it is a known property that:
% 为分析收敛性,我们考虑各项的绝对值。对于任意 $x \ne 0$,存在一个已知性质:
\[
|\tan^{-1}x| < |x|
\]
Applying this property to the iteration for any $x_n \ne 0$ gives a strictly \textbf{contractive} relationship.
% 将此性质应用于迭代过程,对于任意 $x_n \ne 0$ 可得一个严格的压缩关系。
\[
|x_{n+1}| = |\tan^{-1}x_n| < |x_n|
\]
This shows that the sequence of absolute values, $\{|x_n|\}$, is \textbf{strictly decreasing} and \textbf{bounded below by 0}.
% 这表明绝对值序列 $\{|x_n|\}$ 是严格单调递减且有下界0的。
Therefore, the sequence of absolute values must converge to 0.
% 因此,该绝对值序列必然收敛于0。
\[
\lim_{n \to \infty} |x_n| = 0
\]
This implies that the sequence $\{x_n\}$ itself converges to 0 for all initial values $x_0 \in (-\pi/2, \pi/2)$.
% 这意味着序列 $\{x_n\}$ 本身对于所有初始值 $x_0 \in (-\pi/2, \pi/2)$ 都收敛于0。

% 6666666666666666666666

\section*{VI. Analysis of a Continued Fraction}
% VI. 连分数分析

Assuming the sequence converges to a value $x$, this value must be a fixed point satisfying the relation:
% 假设序列收敛于值 $x$,则该值必须是满足以下关系的不动点:
\[
x_n = \frac{1}{p + \frac{1}{p + \dots}} \implies x = \frac{1}{p+x}
\]
The solutions are trivial.
% 使用二次公式可求得其解。
\[
x = \frac{-p \pm \sqrt{p^2 + 4}}{2}
\]
Since the sequence of convergents $x_n$ is always positive for $p > 1$, its limit $x$ must be positive. We thus select the positive root.
% 由于当 $p > 1$ 时,收敛序列 $x_n$ 恒为正,其极限 $x$ 必为正。因此我们选取正根。
\[
x = \frac{-p + \sqrt{p^2+4}}{2}
\]

To prove convergence, we analyze the iteration $x_{n+1} = g(x_n)$ with the function $g(x) = 1/(p+x)$.
% 为证明收敛性,我们分析迭代式 $x_{n+1} = g(x_n)$,其中函数为 $g(x) = 1/(p+x)$。
We will show that $g(x)$ is a contraction mapping for $x \ge 0$. First, we find the derivative.
% 我们将证明 $g(x)$ 在 $x \ge 0$ 上是一个压缩映射。首先计算其导数。
\[
g'(x) = -\frac{1}{(p+x)^2}
\]
The magnitude of the derivative is \textbf{bounded}.For $p > 1$ and any $x \ge 0$, the denominator $(p+x)^2 > p^2$.
% 对于 $p > 1$ 和任意 $x \ge 0$,分母 $(p+x)^2 > p^2$。
This provides a uniform bound on the magnitude of the derivative.
% 这为导数的绝对值提供了一个一致上界。

% 该导数的绝对值有界。
\[
|g'(x)| = \frac{1}{(p+x)^2} < \frac{1}{p^2}
\]

Since $p>1$, we have $p^2 > 1$, which implies that  $1/p^2 < 1$ and $k = \frac{1}{p^2} < 1$.
% 因为 $p>1$,所以 $p^2 > 1$,这意味着上界 $1/p^2$ 小于1。
Because $|g'(x)| \le k < 1$ for all $x \ge 0$, the function $g(x)$ is a contraction mapping on $[0, \infty)$.
% 因为对于所有 $x \ge 0$ 都有 $|g'(x)| \le k < 1$,函数 $g(x)$ 是在 $[0, \infty)$ 上的一个压缩映射。
Therefore, by the \textbf{Contraction Mapping Theorem}, the sequence converges for any initial $x_0 \ge 0$.
% 因此,根据压缩映射定理,该迭代对于任意初始值 $x_0 \ge 0$ 均收敛。
\section*{VII. Bisection Method Analysis When Initial Interval Contains Zero}
% 七、二分法在初始区间包含零点时的分析

\subsection*{Analysis of the Relative Error Measure}
% 相对误差度量的分析

Using the symbols from Problem II, the relative error $\epsilon_n$ and the absolute error bound $|\Delta\alpha|$ are defined by:
% 沿用问题 II 的符号,相对误差 $\epsilon_n$ 和绝对误差 $|\Delta\alpha|$ 的定义为:
\[
\epsilon_n = \left|\frac{x_n - \alpha}{\alpha}\right| = \frac{|\Delta \alpha|}{|\alpha|}
\]

We know that the absolute error is bounded by:
% 我们知道绝对误差的上界为:
\[
|\Delta \alpha| \leq \frac{|b_0 - a_0|}{2^{n+1}}, \quad e_n \leq \epsilon
\]

To guarantee the relative error $\epsilon_n$ is no greater than $\epsilon$ ($\epsilon_n \leq \epsilon$):
% 为保证相对误差 $\epsilon_n$ 不超过 $\epsilon$:
\[
\frac{|\Delta \alpha|}{|\alpha|} \leq \frac{|b_0 - a_0|}{2^{n+1}|\alpha|} \leq \epsilon
\]

This inequality can be rearranged to find the required number of steps $n$, implying that:
% 对此不等式进行整理,可得到所需迭代步数 $n$ 的下限:
\[
n \geq \frac{\log(|b_0 - a_0|) - \log \epsilon - \log |\alpha|}{\log 2} - 1
\]

This result implies that the number of steps $n$ explicitly depends on the magnitude of the root $|\alpha|$, so the relative error is not a good measure in this case.
% 这个结果表明所需的步数 $n$ 显式地依赖于根 $|\alpha|$ 的大小,因此在这种情况下相对误差不是一个好的度量。
Especially when $\alpha$ is around $0$, the relative error is $\infty$ in the limit case $\alpha \to 0$.
% 特别是当 $\alpha$ 接近 $0$ 时,在 $\alpha \to 0$ 的极限情况下相对误差将趋于无穷。
Instead, the absolute error $e^{abs}_n = |x_n - \alpha|$ is a good stand-in.
% 因此,绝对误差 $e^{abs}_n = |x_n - \alpha|$ 是一个恰当的替代度量。

\subsection*{Derivation of the Absolute Error Inequality}
% 绝对误差不等式的推导

The absolute error is bounded by half the interval length after $n$ steps:
% 绝对误差的上界为第 $n$ 步后区间长度的一半:
\[
e^{abs}_n = |\alpha - c_n| \leq \frac{b_n - a_n}{2} = \frac{b_0 - a_0}{2^{n+1}}
\]

To ensure the absolute error $e^{abs}_n \le \epsilon_{abs}$, we require:
% 为确保绝对误差 $e^{abs}_n \le \epsilon_{abs}$,我们要求:
\[
\frac{b_0 - a_0}{2^{n+1}} \leq \epsilon_{abs}
\]

Solving for $n$ by taking the logarithm yields:
% 通过取对数解出 $n$:
\[
n \geq \frac{\log(b_0 - a_0) - \log(\epsilon_{abs})}{\log 2} - 1
\]




\section*{VIII. Modified Newton's Method for Multiple Roots}
% 八、修正牛顿法处理多重根
A root $\alpha$ is considered a root of multiplicity $k$ for a function $f(x) \in C^{k+1}$ if the function can be expressed in the form:

\begin{equation}
    f(x) = (x-\alpha)^k g(x), \quad \text{with } g(x) \in C^1, g(\alpha) \ne 0. \tag{14}+
\end{equation}
\subsection*{Detection of a Multiple Zero}
Considering the iterative process, we have
\begin{align}
  &x_{n+1} - \alpha = x_n - \alpha - \frac{f(x_n)}{f'(x_n)} \nonumber \\
  &e_{n+1} \approx e_n \frac{(k-1) g(x_n)}{k g(x_n) + (x_n - \alpha) g'(x_n)} \nonumber \\
  \implies &\lim_{n \to \infty} \frac{e_{n+1}}{e_n} = \frac{k-1}{k} < 1, \tag{15}
\end{align}

The behavior indicated by (15) can be analyzed more rigorously. By the definition of a root of multiplicity $k$, Taylor's theorem for $f(x)$ and $f'(x)$ around $\alpha$ gives:
$$
f(x_n) = \frac{f^{(k)}(\alpha)}{k!}(x_n-\alpha)^k + O\left((x_n-\alpha)^{k+1}\right)
$$
$$
f'(x_n) = \frac{f^{(k)}(\alpha)}{(k-1)!}(x_n-\alpha)^{k-1} + O\left((x_n-\alpha)^{k}\right)
$$
From these expansions, a primary detection criterion emerges. For a multiple root ($k \ge 2$), it is evident that as $x_n \to \alpha$, both $f(x_n) \to 0$ and $f'(x_n) \to 0$. This is in direct contrast to a simple root ($k=1$), for which $f'(\alpha) \ne 0$ and thus $\lim_{n\to\infty} f'(x_n) \ne 0$.

Furthermore, the ratio in the Newton's step can be approximated using the leading terms of the expansions:
$$
\frac{f(x_n)}{f'(x_n)} \approx \frac{\frac{f^{(k)}(\alpha)}{k!}(x_n-\alpha)^k}{\frac{f^{(k)}(\alpha)}{(k-1)!}(x_n-\alpha)^{k-1}} = \frac{(k-1)!}{k!}(x_n-\alpha) = \frac{1}{k}(x_n-\alpha) = \frac{1}{k}e_n.
$$
Substituting this directly into the error evolution formula provides a concise derivation for the linear convergence:
\begin{align*}
e_{n+1} &= e_n - \frac{f(x_n)}{f'(x_n)} \\
&\approx e_n - \frac{1}{k}e_n = \left(1 - \frac{1}{k}\right)e_n = \left(\frac{k-1}{k}\right)e_n.
\end{align*}
This rigorously confirms the result in (15) and shows that the observed linear convergence is a direct consequence of the fact that $f'(x_n)$ approaches zero at a rate precisely one order lower than $f(x_n)$.

\subsection*{Restoring Quadratic Convergence}
To restore the quadratic rate of convergence, we can apply a modified Newton's method. The analysis of this modified process is as follows:
\begin{align*}
x_{n+1} - \alpha &= x_n - k\frac{f(x_n)}{f'(x_n)} - \alpha \\
&= x_n - k\frac{(x_n - \alpha)g(x_n)}{(x_n - \alpha)g'(x_n) + kg(x_n)} - \alpha\\
&= (x_n - \alpha)\left[1 - \frac{kg(x_n)}{(x_n - \alpha)g'(x_n) + kg(x_n)}\right] \\
&= (x_n - \alpha)^2  g'(x_n)  \frac{1}{(x_n - \alpha)g'(x_n) + kg(x_n)},
\end{align*}
This leads to the following limit for the error ratio, confirming that quadratic convergence is recovered:
$$
\Rightarrow \lim_{n\to\infty} \frac{|e_{n+1}|}{e_n^2} = \lim_{n\to\infty} \left|\frac{g'(x_n)}{(x_n - \alpha)g'(x_n) + kg(x_n)}\right| = \left|\frac{g'(\alpha)}{kg(\alpha)}\right|.
$$





% ===============================================
\section*{ \center{\normalsize {Acknowledgement}} }
Honestly, the \textbf{translation} was done using LLM tools, but the answers and solutions were written by myself.
This version is edited out on 9th Oct.
%20250919


%\printbibliography


\end{document}