\documentclass[a4paper]{article}
\usepackage[affil-it]{authblk}
\usepackage[backend=bibtex,style=numeric]{biblatex}
\usepackage{amsmath}
\usepackage{amssymb}
\usepackage{geometry}
\geometry{margin=1.5cm, vmargin={0pt,1cm}}
\setlength{\topmargin}{-1cm}
\setlength{\paperheight}{29.7cm}
\setlength{\textheight}{25.3cm}

\addbibresource{citation.bib}

\begin{document}
% =================================================
\title{Numerical Analysis homework \# 2}
% 第2次
\author{Wang Hengning 3230104148
  \thanks{Electronic address: \texttt{whning@zju.edu.cn}}}
\affil{Computational mathematics 2301, Zhejiang University }


\date{Due time: \today}

\maketitle

%\begin{abstract}
%    The abstract is not necessary for the theoretical homework, 
%    but for the programming project, 
%    you are encouraged to write one.      
%\end{abstract}


% ============================================
% \section*{I. Briefly repeat the problem}

\section*{I. Analysis of the Linear Interpolation for $\frac{1}{x}$}
% I. 线性插值余项分析

\subsection*{i) Determining $\xi(x)$ Explicitly}
% i) 明确确定 $\xi(x)$ 的表达式

The function values at the nodes are $f(x_0) = f(1) = 1$ and $f(x_1) = f(2) = \frac{1}{2}$.
% 节点处的函数值为 $f(x_0) = f(1) = 1$ 和 $f(x_1) = f(2) = \frac{1}{2}$。

\[
p_1(f; x) = f(x_0)\frac{x-x_1}{x_0-x_1} + f(x_1)\frac{x-x_0}{x_1-x_0}
\]

\[
p_1(f; x) = -(x-2) + \frac{1}{2}(x-1) = -\frac{1}{2}x + \frac{3}{2}
\]
\[
R_1(x) = \frac{1}{x} - \left(-\frac{1}{2}x + \frac{3}{2}\right) = \frac{(x-1)(x-2)}{2x}
\]

\[
f''(x) = 2x^{-3} = \frac{2}{x^3}
\]

\[
\frac{(x-1)(x-2)}{2x} = \frac{f''(\xi(x))}{2}(x - 1)(x - 2)
\]
Since $x \in (1, 2)$, the term $(x-1)(x-2)$ is non-zero, allowing us to simplify the equation:
% 由于 $x \in (1, 2)$,项 $(x-1)(x-2)$ 非零,我们可以简化该方程:
\[
\frac{1}{2x} = \frac{f''(\xi(x))}{2} \implies f''(\xi(x)) = \frac{1}{x}
\]
Substituting the expression for the second derivative, $f''(\xi(x)) = \frac{2}{(\xi(x))^3}$:
% 代入二阶导数的表达式 $f''(\xi(x)) = \frac{2}{(\xi(x))^3}$:
\[
\frac{2}{(\xi(x))^3} = \frac{1}{x} \implies (\xi(x))^3 = 2x
\]
Thus:
\[
\xi(x) = \sqrt[3]{2x}, \quad x \in (1, 2)
\]

\subsection*{ii) Extension and Calculation of Extrema}
% ii) 延拓与极值计算
Since $\xi(x)$ is a \textbf{monotonically} increasing function on $[1, 2]$, its minimum and maximum values occur at the endpoints,  its maximum value occurs at the left endpoint $x=1$:
% 由于 $\xi(x)$ 在 $[1, 2]$ 上是单调递增函数,其最小值和最大值发生在端点:
\[
\min_{x \in [1, 2]} \xi(x) = \xi(1) = \sqrt[3]{2 \cdot 1} = \sqrt[3]{2}
\]
\[
\max_{x \in [1, 2]} \xi(x) = \xi(2) = \sqrt[3]{2 \cdot 2} = \sqrt[3]{4}
\]
\[
\max_{x \in [1, 2]} f''(\xi(x)) = \max_{x \in [1, 2]} \left(\frac{1}{x}\right) = \frac{1}{1} = 1
\]




\section*{II. Construction of a Non-negative Interpolating Polynomial}
% II. 非负插值多项式的构造

We are tasked with finding a polynomial $p \in \mathbb{P}_{2n}^+$ such that it interpolates the given data points $(x_i, f_i)$ for $i=0, 1, \dots, n$, where $f_i \ge 0$.
% 我们的任务是找到一个多项式 $p \in \mathbb{P}_{2n}^+$,使其能够对给定的数据点 $(x_i, f_i)$(其中 $i=0, 1, \dots, n$ 且 $f_i \ge 0$)进行插值。

Let $l_k(x)$ for $k=0, 1, \dots, n$ be the Lagrange basis polynomials corresponding to the distinct nodes $\{x_j\}_{j=0}^n$. These polynomials satisfy the property $l_k(x_j) = \delta_{kj}$.
% 令 $l_k(x)$(其中 $k=0, 1, \dots, n$)为与不同节点 $\{x_j\}_{j=0}^n$ 相对应的拉格朗日基本多项式。这些多项式满足性质 $l_k(x_j) = \delta_{kj}$。

\subsection*{First Construction}
% 第一种构造方法

Consider the polynomial defined as a sum of squares:
% 考虑定义为平方和的多项式:
\[
p(x) = \sum_{k=0}^{n} f_k l_k^2(x)
\]
The degree of each basis polynomial $l_k(x)$ is $n$, which implies the degree of $l_k^2(x)$ is $2n$. Consequently, the degree of $p(x)$ is at most $2n$.
% 每个基本多项式 $l_k(x)$ 的次数为 $n$,这意味着 $l_k^2(x)$ 的次数为 $2n$。因此,$p(x)$ 的次数至多为 $2n$。

Given that $f_k \ge 0$ and $l_k^2(x) \ge 0$ for any real $x$, the sum $p(x)$ is non-negative for all $x \in \mathbb{R}$. Thus, $p(x) \in \mathbb{P}_{2n}^+$.
% 考虑到对于任意实数 $x$ 都有 $f_k \ge 0$ 和 $l_k^2(x) \ge 0$,该求和多项式 $p(x)$ 对所有 $x \in \mathbb{R}$ 都是非负的。因此,$p(x) \in \mathbb{P}_{2n}^+$。

Verifying the interpolation conditions at the nodes $x_i$:
% 在节点 $x_i$ 处验证插值条件:
\[
p(x_i) = \sum_{k=0}^{n} f_k l_k^2(x_i) = f_i l_i^2(x_i) + \sum_{k \neq i} f_k l_k^2(x_i) = f_i (1)^2 + \sum_{k \neq i} f_k (0)^2 = f_i
\]
This construction satisfies all requirements.
% 此构造满足所有要求。

\subsection*{Second Construction}
% 第二种构造方法

Alternatively, consider the polynomial formed by squaring a sum:
% 作为替代,考虑通过对一个和式进行平方来构造多项式:
\[
p(x) = \left( \sum_{k=0}^{n} \sqrt{f_k} l_k(x) \right)^2
\]
Let $S(x) = \sum_{k=0}^{n} \sqrt{f_k} l_k(x)$. Since each $l_k(x)$ is of degree $n$, the degree of $S(x)$ is at most $n$. Therefore, the degree of $p(x) = [S(x)]^2$ is at most $2n$.
% 令 $S(x) = \sum_{k=0}^{n} \sqrt{f_k} l_k(x)$。由于每个 $l_k(x)$ 的次数为 $n$,所以 $S(x)$ 的次数至多为 $n$。因此,$p(x) = [S(x)]^2$ 的次数至多为 $2n$。

The squared form inherently ensures that $p(x) \ge 0$ for all real $x$, so $p(x) \in \mathbb{P}_{2n}^+$.
% 其平方形式内在地保证了对所有实数 $x$ 都有 $p(x) \ge 0$,所以 $p(x) \in \mathbb{P}_{2n}^+$。

Evaluating at the interpolation nodes $x_i$:
% 在插值节点 $x_i$ 处求值:
\[
p(x_i) = \left( \sum_{k=0}^{n} \sqrt{f_k} l_k(x_i) \right)^2 = \left( \sqrt{f_i} l_i(x_i) \right)^2 = \left( \sqrt{f_i} \cdot 1 \right)^2 = f_i
\]
This alternative also fulfills all the problem's conditions.
% 这个替代方案同样满足题目的所有条件。




\section*{III. Analysis of Divided Differences for $f(x)=e^x$}
% III. 函数 $f(x)=e^x$ 的均差分析

\subsection*{Proof by Induction}
% 归纳证明

Let the statement be $P(n): f[t, t+1, \dots, t+n] = \frac{(e-1)^n}{n!}e^t$.
% 设命题为 $P(n): f[t, t+1, \dots, t+n] = \frac{(e-1)^n}{n!}e^t$。

The base case $n=0$ is trivial, as $f[t] = e^t = \frac{(e-1)^0}{0!}e^t$.
% 基础情形 $n=0$ 是显然的,因为 $f[t] = e^t = \frac{(e-1)^0}{0!}e^t$。

Assume $P(n)$ holds for some $n \ge 0$. For $n+1$, the recursive definition of divided differences is:
% 假设 $P(n)$ 对某个 $n \ge 0$ 成立。对于 $n+1$,均差的递归定义为:
\[
f[t, \dots, t+n+1] = \frac{f[t+1, \dots, t+n+1] - f[t, \dots, t+n]}{n+1}
\]
By the inductive hypothesis, the numerator is:
% 根据归纳假设,分子为:
\[
\frac{(e-1)^n}{n!}e^{t+1} - \frac{(e-1)^n}{n!}e^t = \frac{(e-1)^n}{n!}e^t(e-1) = \frac{(e-1)^{n+1}}{n!}e^t
\]
Substituting this back confirms $P(n+1)$:
% 将此代回可证实 $P(n+1)$:
\[
f[t, \dots, t+n+1] = \frac{1}{n+1} \left( \frac{(e-1)^{n+1}}{n!}e^t \right) = \frac{(e-1)^{n+1}}{(n+1)!}e^t
\]
Thus, the formula holds for all integers $n \ge 0$.
% 因此,该公式对所有整数 $n \ge 0$ 均成立。

\subsection*{Determination of $\xi$}
% $\xi$ 的确定

From the result above, setting $t=0$ yields:
% 根据上述结果,令 $t=0$ 可得:
\[
f[0, 1, \dots, n] = \frac{(e-1)^n}{n!}
\]
Corollary 2.22 provides an alternative expression for some $\xi \in (0, n)$:
% 推论 2.22 提供了另一种表达式,其中 $\xi \in (0, n)$:
\[
f[0, 1, \dots, n] = \frac{f^{(n)}(\xi)}{n!}
\]
For $f(x)=e^x$, the $n$-th derivative is $f^{(n)}(x)=e^x$. Equating the two expressions gives:
% 对于 $f(x)=e^x$,其 $n$ 阶导数为 $f^{(n)}(x)=e^x$。令两个表达式相等可得:
\[
\frac{(e-1)^n}{n!} = \frac{e^\xi}{n!}
\]
Solving for $\xi$ directly gives:
% 直接求解 $\xi$ 可得:
\[
\xi = n \ln(e-1)
\]
To compare $\xi$ with the midpoint $n/2$, we analyze the term $\ln(e-1)$.
% 为将 $\xi$ 与中点 $n/2$ 进行比较,我们分析 $\ln(e-1)$ 这一项。
Since $e \approx 2.718$ and $e^{1/2} = \sqrt{e} \approx 1.648$, we have $e-1 > e^{1/2}$.
% 由于 $e \approx 2.718$ 且 $e^{1/2} = \sqrt{e} \approx 1.648$,我们有 $e-1 > e^{1/2}$。
Applying the natural logarithm to this inequality shows that $\ln(e-1) > 1/2$.
% 对该不等式应用自然对数表明 $\ln(e-1) > 1/2$。
Therefore, the position of $\xi$ is determined by:
% 因此,$\xi$ 的位置由以下关系确定:
\[
\xi = n \ln(e-1) > n \cdot \frac{1}{2} = \frac{n}{2}
\]
The point $\xi$ is located to the right of the midpoint of the interval $(0,n)$.
% 点 $\xi$ 位于区间 $(0,n)$ 中点的右侧。


\section*{IV. Newton Interpolation and Minimum Approximation}
% IV. 牛顿插值与最小值近似

\subsection*{Newton Form of the Interpolating Polynomial}
% 插值多项式的牛顿形式

The given data points are $(0, 5)$, $(1, 3)$, $(3, 5)$, and $(4, 12)$.
% 给定的数据点为 $(0, 5)$、$(1, 3)$、$(3, 5)$ 和 $(4, 12)$。
The divided differences are computed and organized in the following table:
% 计算均差并整理成下表:
\[
\begin{array}{c|cccc}
x_i & f(x_i) & f[x_i, x_{i+1}] & f[x_i, \dots, x_{i+2}] & f[x_i, \dots, x_{i+3}] \\
\hline
0 & 5 & & & \\
& & -2 & & \\
1 & 3 & & 1 & \\
& & 1 & & 1/4 \\
3 & 5 & & 2 & \\
& & 7 & & \\
4 & 12 & & & \\
\end{array}
\]
The Newton form of the interpolating polynomial $p_3(x)$ is constructed using the top diagonal of the table.
% 插值多项式 $p_3(x)$ 的牛顿形式由表格的上对角线元素构造。
\[
p_3(x) = f[x_0] + f[x_0, x_1](x-x_0) + f[x_0, x_1, x_2](x-x_0)(x-x_1) + f[x_0, x_1, x_2, x_3](x-x_0)(x-x_1)(x-x_2)
\]
Substituting the coefficients and nodes gives:
% 代入系数和节点可得:
\[
p_3(x) = 5 - 2x + 1 \cdot x(x-1) + \frac{1}{4}x(x-1)(x-3)
\]

\subsection*{Approximation of the Minimum}
% 最小值的近似

To find the minimum, we first find the derivative of the polynomial $p_3(x)$. Expanding the polynomial simplifies differentiation:
% 为了找到最小值,我们首先求多项式 $p_3(x)$ 的导数。展开多项式可以简化求导过程:
\[
p_3(x) = 5 - 2x + (x^2 - x) + \frac{1}{4}(x^3 - 4x^2 + 3x) = \frac{1}{4}x^3 - \frac{9}{4}x + 5
\]
The derivative is:
% 其导数为:
\[
p'_3(x) = \frac{3}{4}x^2 - \frac{9}{4} = \frac{3}{4}(x^2 - 3)
\]
Setting the derivative to zero, $p'_3(x) = 0$, yields the critical points.
% 令导数等于零,$p'_3(x) = 0$,得到临界点。
\[
x^2 - 3 = 0 \implies x = \pm\sqrt{3}
\]
The problem suggests a minimum in the interval $(1,3)$. Since $\sqrt{3} \approx 1.732$ is within this interval, we select it as our candidate.
% 题目暗示在区间 $(1,3)$ 内存在一个最小值。由于 $\sqrt{3} \approx 1.732$ 在此区间内,我们选择它作为候选点。
To confirm it is a minimum, we check the second derivative:
% 为确认其为最小值,我们检查二阶导数:
\[
p''_3(x) = \frac{3}{2}x
\]
At $x = \sqrt{3}$, we have $p''_3(\sqrt{3}) = \frac{3\sqrt{3}}{2} > 0$, which confirms a local minimum.
% 在 $x = \sqrt{3}$ 处,我们有 $p''_3(\sqrt{3}) = \frac{3\sqrt{3}}{2} > 0$,这证实了该点是一个局部最小值。
Thus, the approximate location of the minimum is:
% 因此,最小值的近似位置是:
\[
x_{\min} \approx \sqrt{3}
\]



\section*{V. Divided Differences with Repeated Nodes for $f(x)=x^7$}
% V. 函数 $f(x)=x^7$ 的带重复节点的均差

\subsection*{Computation of the Divided Difference}
% 均差的计算

To handle the repeated nodes, we require the derivatives of $f(x)=x^7$.
% 为了处理重复节点,我们需要函数 $f(x)=x^7$ 的导数。
\[
f'(x) = 7x^6, \quad f''(x) = 42x^5
\]
The values needed for the divided difference table are:
% 均差表所需的数值如下:
\[
f'(1)=7, \quad f''(1)=42, \quad f'(2)=448
\]
We construct the divided difference table for the nodes $z = \{0, 1, 1, 1, 2, 2\}$. For repeated nodes $z_i=z_{i+k}$, the higher-order differences are defined using derivatives, such as $f[z_i, \dots, z_{i+k}] = f^{(k)}(z_i)/k!$.
% 我们为节点序列 $z = \{0, 1, 1, 1, 2, 2\}$ 构建均差表。对于重复节点 $z_i=z_{i+k}$,高阶均差由导数定义,例如 $f[z_i, \dots, z_{i+k}] = f^{(k)}(z_i)/k!$。
\[
\begin{array}{c|cccccc}
z_i & f[z_i] & \text{1st ord.} & \text{2nd ord.} & \text{3rd ord.} & \text{4th ord.} & \text{5th ord.} \\
\hline
0 & 0 & & & & & \\
& & 1 & & & & \\
1 & 1 & & 6 & & & \\
& & f'(1)=7 & & 15 & & \\
1 & 1 & & f''(1)/2=21 & & 42 & \\
& & f'(1)=7 & & 99 & & 30 \\
1 & 1 & & 120 & & 102 & \\
& & 127 & & 201 & & \\
2 & 128 & & 321 & & & \\
& & f'(2)=448 & & & & \\
2 & 128 & & & & & \\
\end{array}
\]
The final entry in the table gives the desired value.
% 表格中的最后一项即为所求值。
\[
f[0, 1, 1, 1, 2, 2] = 30
\]

\subsection*{Determination of $\xi$}
% $\xi$ 的确定

The Mean Value Theorem for divided differences relates the divided difference to a derivative.
% 均差中值定理将均差与导数联系起来。
\[
f[z_0, \dots, z_n] = \frac{f^{(n)}(\xi)}{n!} \quad \text{for some } \xi \in (\min(z_i), \max(z_i))
\]
For this problem, we have $n=5$ and the nodes are within the interval $(0,2)$. The 5th derivative of $f(x)$ is calculated as:
% 对于本题,$n=5$,且节点位于区间 $(0,2)$ 内。函数 $f(x)$ 的五阶导数计算如下:
\[
f^{(5)}(x) = (7 \cdot 6 \cdot 5 \cdot 4 \cdot 3) x^2 = 2520x^2
\]
By substituting the computed divided difference and $n=5$ into the theorem, we get an equation for $\xi$.
% 将计算出的均差值和 $n=5$ 代入定理,我们得到一个关于 $\xi$ 的方程。
\[
30 = \frac{f^{(5)}(\xi)}{5!} = \frac{2520\xi^2}{120}
\]
We solve this equation for $\xi$:
% 我们求解这个关于 $\xi$ 的方程:
\[
\xi^2 = \frac{30 \cdot 120}{2520} = \frac{3600}{2520} = \frac{10}{7}
\]
Since $\xi$ must be in $(0,2)$, we take the positive square root.
% 由于 $\xi$ 必须在 $(0,2)$ 内,我们取正平方根。
\[
\xi = \sqrt{\frac{10}{7}} = \frac{\sqrt{70}}{7}
\]

\section*{VI. Hermite Interpolation and Error Analysis}
% VI. Hermite 插值及误差分析

\subsection*{Estimate of $f(2)$}
% $f(2)$ 的估计

The Hermite interpolation problem is defined by the nodes $z = \{0, 1, 1, 3, 3\}$. We use the divided difference method to find the interpolating polynomial $H_4(x)$.
% Hermite 插值问题由节点序列 $z = \{0, 1, 1, 3, 3\}$ 定义。我们使用均差法来寻找插值多项式 $H_4(x)$。
The required derivative values are $f'(1)=-1$ and $f'(3)=0$. The divided difference table is constructed as follows:
% 所需的导数值为 $f'(1)=-1$ 和 $f'(3)=0$。均差表构造如下:
\[
\begin{array}{c|ccccc}
z_i & f[z_i] & \text{1st} & \text{2nd} & \text{3rd} & \text{4th} \\
\hline
0 & 1 & & & & \\
& & 1 & & & \\
1 & 2 & & -2 & & \\
& & -1 & & 2/3 & \\
1 & 2 & & 0 & & -5/36 \\
& & -1 & & 1/4 & \\
3 & 0 & & 1/2 & & \\
& & 0 & & & \\
3 & 0 & & & & \\
\end{array}
\]
The coefficients from the top diagonal are $1, 1, -2, 2/3, -5/36$. The Newton form of the polynomial is:
% 从上对角线得到的系数为 $1, 1, -2, 2/3, -5/36$。多项式的牛顿形式为:
\[
H_4(x) = 1 + x - 2x(x-1) + \frac{2}{3}x(x-1)^2 - \frac{5}{36}x(x-1)^2(x-3)
\]
We estimate $f(2)$ by evaluating $H_4(2)$.
% 我们通过计算 $H_4(2)$ 来估计 $f(2)$。
\[
f(2) \approx H_4(2) = 1 + 2 - 2(2)(1) + \frac{2}{3}(2)(1)^2 - \frac{5}{36}(2)(1)^2(2-3)
\]
\[
H_4(2) = 3 - 4 + \frac{4}{3} - \frac{5}{36}(-2) = -1 + \frac{4}{3} + \frac{10}{36} = -1 + \frac{24}{18} + \frac{5}{18} = \frac{-18+24+5}{18} = \frac{11}{18}
\]

\subsection*{Maximum Possible Error}
% 最大可能误差

The error in Hermite interpolation for a function $f \in C^5[0,3]$ is given by the formula:
% 对于函数 $f \in C^5[0,3]$,Hermite 插值的误差由以下公式给出:
\[
E_4(x) = f(x) - H_4(x) = \frac{f^{(5)}(\xi)}{5!} \prod_{i=0}^{4} (x-z_i)
\]
where $\xi$ is in the interval spanned by the nodes, $(0,3)$.
% 其中 $\xi$ 位于节点所张成的区间 $(0,3)$ 内。
We want to bound the error at $x=2$. The node product at this point is:
% 我们希望界定在 $x=2$ 处的误差。该点的节点乘积为:
\[
\Psi(2) = (2-0)(2-1)(2-1)(2-3)(2-3) = (2)(1)(1)(-1)(-1) = 2
\]
The error at $x=2$ is therefore:
% 因此,在 $x=2$ 处的误差为:
\[
|E_4(2)| = \left| \frac{f^{(5)}(\xi)}{5!} \cdot \Psi(2) \right| = \left| \frac{f^{(5)}(\xi)}{120} \cdot 2 \right| = \frac{|f^{(5)}(\xi)|}{60}
\]
Given that $|f^{(5)}(x)| \le M$ on $[0,3]$, we can bound the error.
% 给定在 $[0,3]$ 上 $|f^{(5)}(x)| \le M$,我们可以界定该误差。
\[
|E_4(2)| \le \frac{M}{60}
\]
The maximum possible error for the estimate $f(2) \approx 11/18$ is $M/60$.
% 对于估计值 $f(2) \approx 11/18$,其最大可能误差为 $M/60$。


\section*{VII. Relation Between Finite and Divided Differences}
% VII. 有限差分与均差之间的关系

\subsection*{Forward Difference}
% 前向差分

We prove the identity $\Delta^k f(x) = k!h^k f[x_0, x_1, \dots, x_k]$ by induction on $k$, where $x_j = x+jh$.
% 我们通过对 $k$ 进行归纳来证明等式 $\Delta^k f(x) = k!h^k f[x_0, x_1, \dots, x_k]$,其中 $x_j = x+jh$。
For the base case $k=1$, we use the definitions of the operators.
% 对于基础情形 $k=1$,我们使用算子的定义。
\[
\Delta f(x) = f(x+h) - f(x) = f(x_1) - f(x_0)
\]
The right-hand side is:
% 等式右侧为:
\[
1!h^1 f[x_0, x_1] = h \frac{f(x_1)-f(x_0)}{x_1 - x_0} = h \frac{f(x_1)-f(x_0)}{h} = f(x_1)-f(x_0)
\]
The identity holds for $k=1$.
% 该等式对 $k=1$ 成立。
Now, assume the proposition is true for an integer $k \ge 1$. We examine the case for $k+1$.
% 现在,假设命题对某个整数 $k \ge 1$ 成立。我们来考察 $k+1$ 的情形。
\[
\Delta^{k+1}f(x) = \Delta^k f(x+h) - \Delta^k f(x)
\]
Applying the inductive hypothesis to both terms on the right-hand side yields:
% 将归纳假设应用于右侧的两项可得:
\[
\Delta^{k+1}f(x) = k!h^k f[x_1, \dots, x_{k+1}] - k!h^k f[x_0, \dots, x_k] = k!h^k \left( f[x_1, \dots, x_{k+1}] - f[x_0, \dots, x_k] \right)
\]
Using the recursive property of divided differences, we have:
% 利用均差的递归性质,我们有:
\[
f[x_0, \dots, x_{k+1}] = \frac{f[x_1, \dots, x_{k+1}] - f[x_0, \dots, x_k]}{x_{k+1}-x_0}
\]
Since $x_{k+1}-x_0 = (x+(k+1)h) - x = (k+1)h$, we can write:
% 由于 $x_{k+1}-x_0 = (x+(k+1)h) - x = (k+1)h$,我们可以写出:
\[
\Delta^{k+1}f(x) = k!h^k \left( (x_{k+1}-x_0) f[x_0, \dots, x_{k+1}] \right) = k!h^k (k+1)h f[x_0, \dots, x_{k+1}]
\]
This simplifies to the desired result for $k+1$.
% 这可以简化为 $k+1$ 情形下的期望结果。
\[
\Delta^{k+1}f(x) = (k+1)!h^{k+1}f[x_0, \dots, x_{k+1}]
\]
By the principle of induction, the formula is proven for all $k \ge 1$.
% 根据数学归纳法原理,该公式对所有 $k \ge 1$ 都已得证。

\subsection*{Backward Difference}
% 后向差分

A similar proof by induction can be constructed. Alternatively, we can use the established relationship between the backward and forward difference operators, $\nabla f(x) = \Delta f(x-h)$.
% 类似的归纳证明也可以构建。作为替代,我们可以利用后向和前向差分算子之间的既定关系 $\nabla f(x) = \Delta f(x-h)$。
By repeated application, this leads to the identity $\nabla^k f(x) = \Delta^k f(x-kh)$.
% 通过重复应用,可得到恒等式 $\nabla^k f(x) = \Delta^k f(x-kh)$。
We apply the proven forward difference formula to the function evaluated at the point $z = x-kh$:
% 我们将已证明的前向差分公式应用于在点 $z = x-kh$ 处求值的函数:
\[
\Delta^k f(z) = k!h^k f[z, z+h, \dots, z+kh]
\]
Substituting $z=x-kh$ into the nodes gives the set:
% 将 $z=x-kh$ 代入节点集,得到:
\[
\{x-kh, (x-kh)+h, \dots, (x-kh)+kh\} = \{x-kh, x-(k-1)h, \dots, x\}
\]
This set of nodes is precisely $\{x_{-k}, x_{-k+1}, \dots, x_0\}$.
% 这个节点集恰好是 $\{x_{-k}, x_{-k+1}, \dots, x_0\}$。
Therefore, we have:
% 因此,我们有:
\[
\nabla^k f(x) = \Delta^k f(x-kh) = k!h^k f[x_{-k}, x_{-k+1}, \dots, x_0]
\]
Since the divided difference is a symmetric function of its arguments, the order of the nodes is irrelevant.
% 由于均差是其参数的对称函数,节点的顺序无关紧要。
\[
\nabla^k f(x) = k!h^k f[x_0, x_{-1}, \dots, x_{-k}]
\]
This completes the proof.
% 证明完毕。



\section*{VIII. Derivative of Divided Differences}
% VIII. 均差的导数

\subsection*{Partial Derivative with respect to $x_0$}
% 关于 $x_0$ 的偏导数

By the definition of the partial derivative, we have:
% 根据偏导数的定义,我们有:
\[
\frac{\partial}{\partial x_0} f[x_0, x_1, \dots, x_n] = \lim_{h \to 0} \frac{f[x_0+h, x_1, \dots, x_n] - f[x_0, x_1, \dots, x_n]}{h}
\]
Due to the symmetry property of divided differences, we can reorder the arguments of the first term in the numerator.
% 利用均差的对称性,我们可以重排分子中第一项的参数顺序。
\[
f[x_0+h, x_1, \dots, x_n] = f[x_1, \dots, x_n, x_0+h]
\]
The expression inside the limit can then be rewritten as:
% 极限内的表达式因此可以被重写为:
\[
\frac{f[x_1, \dots, x_n, x_0+h] - f[x_0, x_1, \dots, x_n]}{(x_0+h) - x_0}
\]
This is precisely the recursive definition for a higher-order divided difference over the nodes $\{x_0, x_1, \dots, x_n, x_0+h\}$.
% 这恰好是基于节点 $\{x_0, x_1, \dots, x_n, x_0+h\}$ 的高阶均差的递归定义。
\[
\frac{\partial}{\partial x_0} f[x_0, \dots, x_n] = \lim_{h \to 0} f[x_0, x_1, \dots, x_n, x_0+h]
\]
Since $f$ is differentiable, the divided difference is a continuous function of its arguments. We can thus evaluate the limit by substitution.
% 由于 $f$ 是可微的,均差是其参数的连续函数。因此我们可以通过代入来求解极限。
\[
\lim_{h \to 0} f[x_0, x_1, \dots, x_n, x_0+h] = f[x_0, x_1, \dots, x_n, x_0]
\]
Using the symmetry property one last time to group the repeated nodes gives the final result.
% 最后一次使用对称性将重复的节点放在一起,得到最终结果。
\[
\frac{\partial}{\partial x_0} f[x_0, \dots, x_n] = f[x_0, x_0, x_1, \dots, x_n]
\]

\subsection*{Generalization for other variables}
% 对其他变量的推广

The divided difference $f[x_0, \dots, x_n]$ is a symmetric function of its arguments. This allows the result to be generalized to the partial derivative with respect to any other variable $x_i$.
% 均差 $f[x_0, \dots, x_n]$ 是其参数的对称函数。这使得结果可以推广到关于任何其他变量 $x_i$ 的偏导数。
We can permute the arguments to place $x_i$ in the first position without changing the function's value.
% 我们可以排列参数将 $x_i$ 置于首位,而函数值保持不变。
\[
f[x_0, \dots, x_i, \dots, x_n] = f[x_i, x_0, \dots, x_{i-1}, x_{i+1}, \dots, x_n]
\]
Let $Z$ be the set of all other variables $\{x_0, \dots, x_{i-1}, x_{i+1}, \dots, x_n\}$. Differentiating $f[x_i, Z]$ with respect to $x_i$ is directly analogous to the case for $x_0$.
% 令 $Z$ 为所有其他变量的集合 $\{x_0, \dots, x_{i-1}, x_{i+1}, \dots, x_n\}$。对 $f[x_i, Z]$ 关于 $x_i$ 求导与对 $x_0$ 的情况完全类似。
\[
\frac{\partial}{\partial x_i} f[x_i, Z] = f[x_i, x_i, Z]
\]
Substituting the set $Z$ back gives the general formula for any $i \in \{0, 1, \dots, n\}$.
% 将集合 $Z$ 代回,即可得到对于任何 $i \in \{0, 1, \dots, n\}$ 的通用公式。
\[
\frac{\partial}{\partial x_i} f[x_0, \dots, x_n] = f[x_i, x_i, x_0, \dots, x_{i-1}, x_{i+1}, \dots, x_n]
\]
This result represents the divided difference over the original set of nodes with the node $x_i$ repeated.
% 此结果代表在原有节点集的基础上,将节点 $x_i$ 重复一次后所构成的均差。


\section*{IX. A Min-Max Polynomial Problem}
% IX. 一个多项式极小化极大问题

Let $P_n(x) = a_0x^n + a_1x^{n-1} + \dots + a_n$. The problem is to determine the value of $\min_{a_1, \dots, a_n} \max_{x \in [a,b]} |P_n(x)|$.
% 设 $P_n(x) = a_0x^n + a_1x^{n-1} + \dots + a_n$。问题在于确定 $\min_{a_1, \dots, a_n} \max_{x \in [a,b]} |P_n(x)|$ 的值。
We can factor out the fixed leading coefficient $|a_0|$:
% 我们可以提出固定的首项系数 $|a_0|$:
\[
\max_{x \in [a,b]} |P_n(x)| = |a_0| \max_{x \in [a,b]} |x^n + \frac{a_1}{a_0}x^{n-1} + \dots + \frac{a_n}{a_0}|
\]
Minimizing over $a_1, \dots, a_n$ is equivalent to minimizing over the coefficients of the resulting monic polynomial. Let $\mathcal{P}_n^M$ be the set of all monic polynomials of degree $n$. The problem is transformed into:
% 对 $a_1, \dots, a_n$ 取极小值等价于对所得的首一多项式的系数取极小值。设 $\mathcal{P}_n^M$ 是所有 $n$ 次首一多项式的集合。该问题可转化为:
\[
|a_0| \min_{p \in \mathcal{P}_n^M} \max_{x \in [a,b]} |p(x)|
\]
We use the linear transformation $x = \frac{b-a}{2}t + \frac{a+b}{2}$ to map the interval $t \in [-1, 1]$ onto $x \in [a,b]$.
% 我们使用线性变换 $x = \frac{b-a}{2}t + \frac{a+b}{2}$ 将区间 $t \in [-1, 1]$ 映射到 $x \in [a,b]$。
A monic polynomial $p(x) \in \mathcal{P}_n^M$ can be expressed in terms of $t$:
% 一个首一多项式 $p(x) \in \mathcal{P}_n^M$ 可以用变量 $t$ 来表示:
\[
p(x) = \left(\frac{b-a}{2}t + \frac{a+b}{2}\right)^n + \dots = \left(\frac{b-a}{2}\right)^n t^n + \text{lower degree terms in } t
\]
This implies that $p(x)$ can be written as $(\frac{b-a}{2})^n q(t)$, where $q(t)$ is a monic polynomial in $t$ of degree $n$.
% 这意味着 $p(x)$ 可以写成 $(\frac{b-a}{2})^n q(t)$ 的形式,其中 $q(t)$ 是一个关于 $t$ 的 $n$ 次首一多项式。
The min-max problem over $[a,b]$ becomes a corresponding problem over $[-1,1]$.
% 在 $[a,b]$ 上的极小化极大问题相应地变成了在 $[-1,1]$ 上的问题。
\[
\min_{p \in \mathcal{P}_n^M} \max_{x \in [a,b]} |p(x)| = \min_{q \in \mathcal{P}_n^M} \max_{t \in [-1,1]} \left|\left(\frac{b-a}{2}\right)^n q(t)\right| = \left(\frac{b-a}{2}\right)^n \min_{q \in \mathcal{P}_n^M} \max_{t \in [-1,1]} |q(t)|
\]
A fundamental theorem of approximation theory states that the minimum value of the maximum absolute value for a monic polynomial of degree $n$ on $[-1,1]$ is achieved by the monic Chebyshev polynomial $\tilde{T}_n(t) = T_n(t)/2^{n-1}$.
% 逼近论中的一个基本定理指出,在 $[-1,1]$ 上,$n$ 次首一多项式的最大绝对值的最小值由首一 Chebyshev 多项式 $\tilde{T}_n(t) = T_n(t)/2^{n-1}$ 达到。
\[
\min_{q \in \mathcal{P}_n^M} \max_{t \in [-1,1]} |q(t)| = \max_{t \in [-1,1]} |\tilde{T}_n(t)| = \frac{1}{2^{n-1}}
\]
Substituting this result back, we obtain the final solution.
% 将此结果代回,我们得到最终解。
\[
\min_{a_1, \dots, a_n} \max_{x \in [a,b]} |P_n(x)| = |a_0| \left(\frac{b-a}{2}\right)^n \frac{1}{2^{n-1}} = |a_0| \frac{(b-a)^n}{2^{2n-1}}
\]


\section*{X. A Minimization Property of Scaled Chebyshev Polynomials}
% X. 缩放的切比雪夫多项式的一个最小化性质

We proceed by contradiction. Assume there exists a polynomial $p(x) \in \mathbb{P}_n^a$ such that $\|p\|_{\infty} < \|\hat{p}_n\|_{\infty}$.
% 我们使用反证法。假设存在一个多项式 $p(x) \in \mathbb{P}_n^a$ 满足 $\|p\|_{\infty} < \|\hat{p}_n\|_{\infty}$。
Define a new polynomial $Q(x) = \hat{p}_n(x) - p(x)$. Since $\hat{p}_n$ and $p$ are both polynomials of degree $n$, $Q(x)$ is a polynomial of degree at most $n$.
% 定义一个新多项式 $Q(x) = \hat{p}_n(x) - p(x)$。由于 $\hat{p}_n$ 和 $p$ 都是 $n$ 次多项式,因此 $Q(x)$ 是一个次数至多为 $n$ 的多项式。

The Chebyshev polynomial $T_n(x)$ attains its extrema on $[-1,1]$ at the $n+1$ points $x_k = \cos(k\pi/n)$ for $k=0,1,\dots,n$, where $T_n(x_k) = (-1)^k$.
% 切比雪夫多项式 $T_n(x)$ 在 $[-1,1]$ 上的 $n+1$ 个极值点为 $x_k = \cos(k\pi/n)$ (其中 $k=0,1,\dots,n$),在这些点上 $T_n(x_k) = (-1)^k$。
The polynomial $\hat{p}_n(x) = T_n(x)/T_n(a)$ therefore satisfies:
% 因此,多项式 $\hat{p}_n(x) = T_n(x)/T_n(a)$ 满足:
\[
\hat{p}_n(x_k) = \frac{(-1)^k}{T_n(a)}
\]
Since $a > 1$, $T_n(a) > 0$. The maximum absolute value of $\hat{p}_n(x)$ on $[-1,1]$ is thus $\|\hat{p}_n\|_{\infty} = 1/T_n(a)$.
% 由于 $a > 1$, $T_n(a) > 0$。因此 $\hat{p}_n(x)$ 在 $[-1,1]$ 上的最大绝对值为 $\|\hat{p}_n\|_{\infty} = 1/T_n(a)$。

At these points $x_k$, we evaluate $Q(x_k)$:
% 在这些点 $x_k$ 上,我们计算 $Q(x_k)$ 的值:
\[
Q(x_k) = \hat{p}_n(x_k) - p(x_k)
\]
From our initial assumption, $|p(x_k)| \le \|p\|_{\infty} < \|\hat{p}_n\|_{\infty} = |\hat{p}_n(x_k)|$.
% 根据我们的初始假设,有 $|p(x_k)| \le \|p\|_{\infty} < \|\hat{p}_n\|_{\infty} = |\hat{p}_n(x_k)|$。
This implies that $p(x_k)$ cannot change the sign of $\hat{p}_n(x_k)$. Thus, for $k=0, \dots, n$:
% 这意味着 $p(x_k)$ 无法改变 $\hat{p}_n(x_k)$ 的符号。因此,对于 $k=0, \dots, n$:
\[
\mathrm{sign}(Q(x_k)) = \mathrm{sign}(\hat{p}_n(x_k)) = \mathrm{sign}((-1)^k)
\]
Since $Q(x)$ alternates in sign across the $n+1$ points $x_k$, by the Intermediate Value Theorem, $Q(x)$ must have at least $n$ distinct roots in the interval $(-1,1)$.
% 由于 $Q(x)$ 在这 $n+1$ 个点 $x_k$ 上符号交替,根据中值定理,$Q(x)$ 在区间 $(-1,1)$ 内必定有至少 $n$ 个不同的根。

Furthermore, both $\hat{p}_n(x)$ and $p(x)$ are in $\mathbb{P}_n^a$, which means $\hat{p}_n(a) = 1$ and $p(a)=1$.
% 此外,$\hat{p}_n(x)$ 和 $p(x)$ 均属于 $\mathbb{P}_n^a$,这意味着 $\hat{p}_n(a) = 1$ 且 $p(a)=1$。
Therefore, $Q(x)$ has an additional root at $x=a$:
% 因此,$Q(x)$ 在 $x=a$ 处有另一个根:
\[
Q(a) = \hat{p}_n(a) - p(a) = 1 - 1 = 0
\]
Since $a > 1$, this root is distinct from the $n$ roots found within $(-1,1)$. This gives $Q(x)$ at least $n+1$ distinct roots.
% 因为 $a > 1$,这个根与在 $(-1,1)$ 内找到的 $n$ 个根都不同。这表明 $Q(x)$ 至少有 $n+1$ 个不同的根。

We have established that $Q(x)$ is a polynomial of degree at most $n$ with at least $n+1$ roots. This is only possible if $Q(x)$ is identically zero.
% 我们已经确定 $Q(x)$ 是一个次数至多为 $n$ 的多项式,却拥有至少 $n+1$ 个根。这唯一的可能性是 $Q(x)$ 为零多项式。
If $Q(x) \equiv 0$, then $p(x) \equiv \hat{p}_n(x)$, which implies $\|p\|_{\infty} = \|\hat{p}_n\|_{\infty}$. This contradicts our initial assumption that $\|p\|_{\infty} < \|\hat{p}_n\|_{\infty}$.
% 如果 $Q(x) \equiv 0$,那么 $p(x) \equiv \hat{p}_n(x)$,这意味着 $\|p\|_{\infty} = \|\hat{p}_n\|_{\infty}$。这与我们最初的假设 $\|p\|_{\infty} < \|\hat{p}_n\|_{\infty}$ 相矛盾。
The assumption must be false, and therefore, for any $p \in \mathbb{P}_n^a$, we must have $\|\hat{p}_n\|_{\infty} \le \|p\|_{\infty}$.
% 假设必为假,因此,对于任何 $p \in \mathbb{P}_n^a$,必有 $\|\hat{p}_n\|_{\infty} \le \|p\|_{\infty}$。


\section*{XI. Proof of the Degree Elevation Property for Bernstein Polynomials}
% XI. 伯恩斯坦多项式升阶性质的证明

The proof begins by expanding the right-hand side (RHS) of the identity using the definition of a Bernstein basis polynomial, $b_{n,k}(t) = \binom{n}{k}t^k(1-t)^{n-k}$.
% 证明从恒等式右侧(RHS)展开开始,使用伯恩斯坦基多项式的定义 $b_{n,k}(t) = \binom{n}{k}t^k(1-t)^{n-k}$。
\[
\text{RHS} = \frac{n-k}{n}b_{n,k}(t) + \frac{k+1}{n}b_{n,k+1}(t)
\]
Substituting the definition yields:
% 代入定义可得:
\[
\text{RHS} = \frac{n-k}{n}\binom{n}{k}t^k(1-t)^{n-k} + \frac{k+1}{n}\binom{n}{k+1}t^{k+1}(1-t)^{n-k-1}
\]
Next, we expand the binomial coefficients and simplify the constant factors.
% 接下来,我们展开二项式系数并简化常数因子。
For the first term's coefficient:
% 对于第一项的系数:
\[
\frac{n-k}{n}\binom{n}{k} = \frac{n-k}{n} \frac{n!}{k!(n-k)!} = \frac{n-k}{n} \frac{n(n-1)!}{k!(n-k)(n-k-1)!} = \frac{(n-1)!}{k!(n-k-1)!} = \binom{n-1}{k}
\]
For the second term's coefficient:
% 对于第二项的系数:
\[
\frac{k+1}{n}\binom{n}{k+1} = \frac{k+1}{n} \frac{n!}{(k+1)!(n-k-1)!} = \frac{k+1}{n} \frac{n(n-1)!}{(k+1)k!(n-k-1)!} = \frac{(n-1)!}{k!(n-k-1)!} = \binom{n-1}{k}
\]
Substituting these simplified coefficients back into the expression for the RHS:
% 将这些简化后的系数代回 RHS 的表达式中:
\[
\text{RHS} = \binom{n-1}{k}t^k(1-t)^{n-k} + \binom{n-1}{k}t^{k+1}(1-t)^{n-k-1}
\]
We can factor out the common terms $\binom{n-1}{k}t^k(1-t)^{n-k-1}$.
% 我们可以提出公因式 $\binom{n-1}{k}t^k(1-t)^{n-k-1}$。
\[
\text{RHS} = \binom{n-1}{k}t^k(1-t)^{n-k-1} \left[ (1-t) + t \right]
\]
This simplifies to:
% 化简后得到:
\[
\text{RHS} = \binom{n-1}{k}t^k(1-t)^{(n-1)-k} = b_{n-1,k}(t)
\]
The final expression is the definition of $b_{n-1,k}(t)$, which is the left-hand side (LHS) of the identity. This completes the proof.
% 最终的表达式即为 $b_{n-1,k}(t)$ 的定义,也就是恒等式的左侧(LHS)。证明完毕。


\section*{XII. Proof of the Integral Property of Bernstein Basis Polynomials}
% XII. 伯恩斯坦基多项式积分性质的证明

Let $I_{n,k}$ denote the integral of the Bernstein basis polynomial $b_{n,k}(t)$ over the interval $[0,1]$.
% 令 $I_{n,k}$ 表示伯恩斯坦基多项式 $b_{n,k}(t)$ 在区间 $[0,1]$ 上的积分。
\[
I_{n,k} = \int_0^1 b_{n,k}(t) dt = \binom{n}{k} \int_0^1 t^k (1-t)^{n-k} dt
\]
We apply integration by parts to the integral, with $u=(1-t)^{n-k}$ and $dv=t^k dt$. This gives $du=-(n-k)(1-t)^{n-k-1}dt$ and $v=t^{k+1}/(k+1)$.
% 我们对该积分使用分部积分法,令 $u=(1-t)^{n-k}$ 且 $dv=t^k dt$。可得 $du=-(n-k)(1-t)^{n-k-1}dt$ 且 $v=t^{k+1}/(k+1)$。
\[
I_{n,k} = \binom{n}{k} \left[ \left. \frac{t^{k+1}(1-t)^{n-k}}{k+1} \right|_0^1 + \frac{n-k}{k+1} \int_0^1 t^{k+1} (1-t)^{n-k-1} dt \right]
\]
The boundary term evaluates to zero for $k < n$. Thus, the expression simplifies to:
% 对于 $k < n$,边界项的值为零。因此,表达式简化为:
\[
I_{n,k} = \frac{n-k}{k+1} \binom{n}{k} \int_0^1 t^{k+1} (1-t)^{n-(k+1)} dt
\]
We analyze the coefficient:
% 我们分析其系数:
\[
\frac{n-k}{k+1} \binom{n}{k} = \frac{n-k}{k+1} \frac{n!}{k!(n-k)!} = \frac{n!}{(k+1)!(n-k-1)!} = \binom{n}{k+1}
\]
Substituting this back, we establish a recurrence relation for the integral.
% 将此代回,我们便为该积分建立了一个递推关系。
\[
I_{n,k} = \binom{n}{k+1} \int_0^1 t^{k+1} (1-t)^{n-(k+1)} dt = I_{n,k+1}
\]
This relation holds for $k = 0, 1, \dots, n-1$, implying that the value of the integral $I_{n,k}$ is constant for all $k$.
% 该关系对 $k = 0, 1, \dots, n-1$ 均成立,意味着积分 $I_{n,k}$ 的值对所有 $k$ 都是一个常数。
To find this constant value, we use the partition of unity property of Bernstein basis polynomials:
% 为求得此常数值,我们利用伯恩斯坦基多项式的单位分解性质:
\[
\sum_{k=0}^n b_{n,k}(t) = 1
\]
Integrating this identity over $[0,1]$ yields:
% 将此恒等式在 $[0,1]$ 上积分可得:
\[
\int_0^1 \sum_{k=0}^n b_{n,k}(t) dt = \sum_{k=0}^n \int_0^1 b_{n,k}(t) dt = \int_0^1 1 dt = 1
\]
Since all $n+1$ integrals in the sum are equal, we have:
% 由于和式中所有 $n+1$ 个积分均相等,我们有:
\[
(n+1) I_{n,k} = 1
\]
This leads to the final result, which is independent of $k$.
% 这便导出了与 $k$ 无关的最终结果。
\[
\int_0^1 b_{n,k}(t) dt = \frac{1}{n+1}
\]












































% ===============================================
\section*{ \center{\normalsize {Acknowledgement}} }
Honestly, the \textbf{translation} was done using LLM tools, but the answers and solutions were written by myself.
This version is edited out on 9th Oct.
%20250919


%\printbibliography


\end{document}